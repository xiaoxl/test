\documentclass{amsart}



    \title{review of ``On topological approach to local theory of surfaces in Calabi-Yau threefolds''}
    \author{Xinli Xiao}
    \date{\today}
    
    \begin{document}
    
    \maketitle
    In this paper the authors explore a series of enumerative invariants and study the inter-relations and dualities between them. In particular the authors focus on Donaldson-Thomas gauge theory on $3$-folds and its reduction to $4$-dimensional and $2$-dimensional case which are relevant to the local theory of surfaces in Calabi-Yau $3$-folds.
    
    The paper consists of five sections. Section $1$ is a quick introduction and Section $2$ is a review of Gromov-Witten theory, Donaldson-Thomas theory and Seiberg-Witten theory. 
    
    In Section $3$ the authors study the case that the $3$-fold $X$ is an elliptic fibration over a surface $S$. In this case the Donaldson-Thomas theory on $X$ can be computed by the Vafa-Witten theory on $S$ through the dimensional reduction process. Two examples are computed explicitly under this setting, and the Donaldson-Thomas theory is compared to the Gromov-Witten theory for both of them. The case of the total space of the canonical line bundle of a Fano surface is also discussed.
    
    In Section $4$ the authors first study the case that $X=\Sigma_{\ell}\times S$ where $\Sigma_{\ell}$ is a Riemann surface of genus $\ell$ and $S$ is a K\"ahler surface. In this case the Donaldson-Thomas theory on $X$ can be reduced to the $4$-dimensional gauge theory on $S$, or to the $\mathcal N=2$ sigma-model on $\Sigma_{\ell}$. The moduli spaces of each theory are discussed and the correlation functions are computed. Then at the end of Section $4$ the authors summarize many theorems and conjectures which describe the relations between various enumerative invariants on various spaces. 
    
    In Section $5$ the authors study the case that $X$ is the total space of the canonical line bundle over a smooth projective surface $S$ which satisfies a specific condition. In this case, a modified Donaldson-Thomas theory of $X$ is computed by the Seiberg-Witten theory on $S$ and some invariants of ``nested Hilbert schemes'' on $S$. The formula is given explicitly.
    
    \end{document}
    